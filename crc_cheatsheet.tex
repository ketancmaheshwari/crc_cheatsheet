%---------------------------------------------------
%	PACKAGES AND OTHER DOCUMENT CONFIGURATIONS
%---------------------------------------------------

\documentclass[landscape,a0paper,fontscale=0.285]{baposter} % Adjust the font scale/size here
\title{Cheat Sheet}
\usepackage[brazilian]{babel}
\usepackage[utf8]{inputenc}

\usepackage{graphicx} % Required for including images
\graphicspath{{figures/}} % Directory in which figures are stored

\usepackage{xcolor}
\usepackage{colortbl}
\usepackage{tabu}
%\usepackage{hyperref}

\usepackage{mathtools}
%\usepackage{amsmath} % For typesetting math
\usepackage{amssymb} % Adds new symbols to be used in math mode

\usepackage{booktabs} % Top and bottom rules for tables
\usepackage{enumitem} % Used to reduce itemize/enumerate spacing
\usepackage{palatino} % Use the Palatino font
\usepackage[font=small,labelfont=bf]{caption} % Required for specifying captions to tables and figures

\usepackage{multicol} % Required for multiple columns
\setlength{\columnsep}{1.5em} % Slightly increase the space between columns
\setlength{\columnseprule}{0mm} % No horizontal rule between columns

\usepackage{tikz} % Required for flow chart
\usetikzlibrary{decorations.pathmorphing}
\usetikzlibrary{shapes,arrows} % Tikz libraries required for the flow chart in the template
\setlist[itemize]{leftmargin=*}
\setlist[enumerate]{leftmargin=*}
\newcommand{\compresslist}{ % Define a command to reduce spacing within itemize/enumerate environments, this is used right after \begin{itemize} or \begin{enumerate}
\setlength{\itemsep}{1pt}
\setlength{\parskip}{0pt}
\setlength{\parsep}{0pt}
}

\definecolor{lightblue}{rgb}{0.145,0.6666,1} % Defines the color used for content box headers
\definecolor{pittblue}{RGB}{20,41,87} % Defines the color used for content box headers
\definecolor{pittgold}{RGB}{205,184,125} % Defines the color used for content box headers

\begin{document}

\begin{poster}
{
 headerborder=closed, % Adds a border around the header of content boxes
 colspacing=0.8em, % Column spacing
 bgColorOne=white, % Background color for the gradient on the left side of the poster
 bgColorTwo=white, % Background color for the gradient on the right side of the poster
 borderColor=pittblue, % Border color
 headerColorOne=pittblue, % Background color for the header in the content boxes (left side)
 headerColorTwo=pittgold, % Background color for the header in the content boxes (right side)
 headerFontColor=white, % Text color for the header text in the content boxes
 boxColorOne=white, % Background color of the content boxes
 textborder=roundedleft, % Format of the border around content boxes, can be: none, bars, coils, triangles, rectangle, rounded, roundedsmall, roundedright or faded
 eyecatcher=true, % Set to false for ignoring the left logo in the title and move the title left
 headerheight=0.1\textheight, % Height of the header
 headershape=roundedright, % Specify the rounded corner in the content box headers, can be: rectangle, small-rounded, roundedright, roundedleft or rounded
 headerfont=\Large\bf\textsc, % Large, bold and sans serif font in the headers of content boxes
 %textfont={\setlength{\parindent}{1.5em}}, % Uncomment for paragraph indentation
 linewidth=2pt % Width of the border lines around content boxes
}
%----------------------------------------------------------------
%	TITLE
%----------------------------------------------------------------
{\bf\textsc{Cheat Sheet}\vspace{0.5em}} % Poster title
{\textsc{C h e a t \ \ \ \ \ S h e e t \hspace{12pt}}}
{\textsc{Center for Research Computing \\ University of Pittsburgh \hspace{12pt}}} 

%------------------------------------------------
% ACCESS
%------------------------------------------------
\headerbox{Access}{name=access,column=0,row=0}{

%--------------------------------------
\colorbox[HTML]{cdb87d}{\makebox[\textwidth-2\fboxsep][l]{\bf - Request an Account}}
\vspace{1 mm}
\texttt{http://core.sam.pitt.edu/apply}
\colorbox[HTML]{cdb87d}{\makebox[\textwidth-2\fboxsep][l]{\bf - Login}}
HTC \; \texttt{htc.sam.pitt.edu} \\
SMP/OPA/NTA\\ \; \texttt{h2p.crc.pitt.edu} OR\\
\;\texttt{login0b.mpi.sam.pitt.edu}
%--------------------------------------
\colorbox[HTML]{cdb87d}{\makebox[\textwidth-2\fboxsep][l]{\bf - Storage}} 
%Storage facilities
ihome (75G per user) \; \texttt{/ihome} \\
mobydisk (4T per \emph{group}) \; \texttt{/mnt/mobydisk} \\
ZFS (5T per \emph{group}) \; \texttt{/zfs1 \& /zfs2}
\begin{itemize}\compresslist
\item your home directory on each of these storage systems is under your primary group name directory. It is backed up.
\item Use the \texttt{id} command to find your primary group.
\item Both mobydisk and ZFS are \textbf{not backed up}. 
\item To get ZFS storage space open a help ticket.
\end{itemize}
%\vspace{1.0em} % When there are two boxes, some whitespace may need to be added if the one on the right has more content
}

%\headerbox{Common Commands}{name=introduction,column=1,row=0,bottomaligned=objectives}{
\headerbox{Common Commands}{name=commoncommands, column=1,span=2, row=0}{
%------
\colorbox[HTML]{cdb87d}{\makebox[\textwidth-2\fboxsep][l]{\bf - Connect}}

\texttt{ssh h2p.crc.pitt.edu}\\
\texttt{ssh htc.sam.pitt.edu}\\
\texttt{ssh login0b.mpi.sam.pitt.edu}\\
%------
\colorbox[HTML]{cdb87d}{\makebox[\textwidth-2\fboxsep][l]{\bf - Data}}
\texttt{scp afile.tgz htc.sam.pitt.edu:\textasciitilde/} \hfill  \# copy from your local computer to cluster \\
\texttt{scp htc.sam.pitt.edu:\textasciitilde/afile.tgz .} \hfill  \# copy from cluster to your local computer \\
\texttt{rsync -aP \$HOME/src /zfs1/1/sam/ketan} \hfill  \# copy src to zfs using rsync\\
\texttt{df -h /zfs1/2/kjordan/} \hfill \# find quota of a group's ZFS storage\\
\texttt{lfs quota -g jpipas /mnt/mobydisk} \hfill \# find quota of a group's mobydisk storage

\colorbox[HTML]{cdb87d}{\makebox[\textwidth-2\fboxsep][l]{\bf - LMod: load and unload software}}
\texttt{module spider fftw} \hfill  \# search for module named fftw \\
\texttt{module avail}  \hfill  \# list available modules \\
\texttt{module list}   \hfill  \# list currently loaded modules \\
\texttt{module load compiler/python/2.7.10-Anaconda-2.3.0}   \hfill  \# load a module named anaconda\\
\texttt{module purge} \hfill  \# unload all modules 
%------
}

%----------------------------------------------------------------
%	Help
%----------------------------------------------------------------
\headerbox{Help}{name=help,column=3,span=1,row=0}{
Does the FAQ answer your question?\\ \texttt{http://core.sam.pitt.edu/faqs}\\
Search the website:\\ \texttt{http://core.sam.pitt.edu/search}\\
Read the documentation\\
\texttt{http://core.sam.pitt.edu/node/6}\\
%Seek help \\ 
%\linebreak
%% {lp{5.8cm}lp{1.0cm}|}
%-----
\colorbox[HTML]{cdb87d}{\makebox[\textwidth-2\fboxsep][l]{\bf - When submitting support ticket}}
\begin{enumerate}\compresslist
\item Provide a descriptive, specific title
\item Specify the cluster the ticket applies to
\item Provide directory location if applicable
\end{enumerate}
\texttt{{\small core.sam.pitt.edu/node/add/support-ticket}}
\linebreak\\
%------
\colorbox[HTML]{cdb87d}{\makebox[\textwidth-2\fboxsep][l]{\bf - CRC Consultants and their Expertise}}
Kim Wong: Bio Simulation | Agent-based Modeling | Physics-based Modeling \\
Fangping Mu: Bioinformatics | Computational Biology | Computational Genomics \\
Barry Moore: Quantum Chemistry | HPC \\
Ketan Maheshwari: GPU Computing | Scientific Workflows  \\
Shervin Sammak: Turbulent Combustion | Fluid Dynamics\\
\colorbox[HTML]{cdb87d}{\makebox[\textwidth-2\fboxsep][l]{\bf - Contact / Feedback}}
Please send your feedback and suggestions for improvement to this document at \texttt{http://core.sam.pitt.edu/contact}
%-----------------------------------
}

%----------------------------------------------------------------
%	Software
%----------------------------------------------------------------
\headerbox{Software}{name=software,column=0,row=1, below=access}{
\colorbox[HTML]{cdb87d}{\makebox[\textwidth-2\fboxsep][l]{\bf - Applications}}
\begin{itemize}\compresslist
\item Chemistry: NAMD$^{s}$, casino, lammps, Amber$^{g}$, {\color{red}Molpro$^{s}$}, {\color{red}Turbomole$^{s}$}, CP2K$^{s}$, {\color{red}Gaussian$^{s}$}
\item Bio: CLCbio$^{h}$, galaxy$^{h}$, bowtie$^{h}$, samtools$^{h}$, picard$^{h}$, trinity$^{h}$
\item Material Sci: {\color{red}Material Studio}, Westpa, {\color{red}abaqus}, {\color{red}VASP$^{s,o}$}
\end{itemize}

%----Libraries and Programming-----------
\colorbox[HTML]{cdb87d}{\makebox[\textwidth-2\fboxsep][l]{\bf - Libraries and Programming}}
\begin{itemize}\compresslist
\item Compilers: GCC$^{o,s,h}$, Java$^{h,s}$, {\color{red}Intel$^{o,s}$}
\item Scripting: Python$^{h,s}$, Perl$^{h,s,o}$, R$^{h,s}$, {\color{red}Matlab}
\item Libraries: Boost$^{h,s}$, FFT$^{s}$, tensorflow$^{h,g}$, mkl$^{s}$, hdf5$^{s}$, bioconductor$^{h}$
\end{itemize}

%-------Others---------------------
\colorbox[HTML]{cdb87d}{\makebox[\textwidth-2\fboxsep][l]{\bf - Others}}
\begin{itemize}\compresslist
\item Editors: vim$^{o,s,h}$, emacs$^{o,s,h}$, nano$^{o,s,h}$, gedit$^{o,s,h}$
\item Debuggers: gdb$^{o,s,h}$, gprof$^{o,s,h}$
\item Shells: bash$^{o,s,h}$, zsh$^{o,s,h}$
\end{itemize}
}

%----------------------------------------------------------------
%	Hardware
%----------------------------------------------------------------
\headerbox{Hardware}{name=hardware,column=1, span=2, row=1, below=commoncommands}{
\colorbox[HTML]{cdb87d}{\makebox[\textwidth-2\fboxsep][l]{\bf - SMP}}
\begin{itemize}\compresslist
\item 24 nodes of 12-core Xeon E5-2643v4 3.40 GHz (Broadwell), 256 G RAM, 256 G SSD \& 1 T SSD, 10 GigE
\item 2 nodes of 12-core Xeon E5-2643v4 3.40 GHz (Broadwell), 256 G RAM, 256 G SSD \& 3 T SSD, 10 GigE
\item 2 nodes of 12-core Xeon E5-2643v4 3.40 GHz (Broadwell), 512 G RAM, 256 G SSD \& 3 T SSD, FDR Infiniband
\item 1 node of 12-core Xeon E5-2643v4 3.40 GHz (Broadwell), 256 G RAM, 256 G SSD \& 6 T NVMe, GigE
\end{itemize}

\colorbox[HTML]{cdb87d}{\makebox[\textwidth-2\fboxsep][l]{\bf - OPA}}
\begin{itemize}\compresslist
%\item (IB) 32 nodes of 20-core Haswell (E5-2660 v3), 2.6 GHz (Haswell), 128 G RAM, 256 G SSD, FDR InfiniBand
\item 96 nodes of 28-core Intel Xeon E5-2690 2.60 GHz (Broadwell), 64 G RAM, 256 G SSD, 100 Gb Omni-Path
\item 8 nodes of 256-core (hyper-threaded) Intel KNL Xeon Phi 7210 1.30 GHz, 94 G RAM
\end{itemize}

\colorbox[HTML]{cdb87d}{\makebox[\textwidth-2\fboxsep][l]{\bf - HTC}}
\begin{itemize}\compresslist
\item 20 nodes of 16-core Intel Xeon E5-2630v3, 2.4GHz (Haswell-EP), 256 G RAM, 256 G SSD, FDR Infiniband
\end{itemize}

\colorbox[HTML]{cdb87d}{\makebox[\textwidth-2\fboxsep][l]{\bf - NTA}}
\begin{itemize}\compresslist
\item 7 nodes with 4 NVIDIA Titan X GPGPUs/node
\item 8 nodes with 4 NVIDIA GeForce GTX 1080 GPGPUs/node
\item 1 node with 2 NVIDIA K40 GPGPUs
\end{itemize}

}
\headerbox{Charging Policy}{name=method,column=3, below=help}{
Users are charged in terms of Service Units (SU) which depend on both memory and CPU usage.
\textbf{Details coming soon.}
\colorbox[HTML]{cdb87d}{\makebox[\textwidth-2\fboxsep][l]{\bf - CPU}}
\linebreak \\
\colorbox[HTML]{cdb87d}{\makebox[\textwidth-2\fboxsep][l]{\bf - Memory}}

}

\end{poster}
\newpage

%%%%%%%%%%%%%%%%%%%%%%%%%%%%%%%%%%%%%%%%%%%%%%%%%%%%%%%%%%
%%%%%%%%%%%%%%%%%%    SECOND PAGE    %%%%%%%%%%%%%%%%%%
%%%%%%%%%%%%%%%%%%%%%%%%%%%%%%%%%%%%%%%%%%%%%%%%%%%%%%%%%%

\begin{poster}
{
borderColor=pittblue, headerColorOne=pittblue, headerColorTwo=pittgold, headerborder=closed, colspacing=0.8em, bgColorOne=white, bgColorTwo=white, headerFontColor=white, boxColorOne=white, textborder=roundedleft, eyecatcher=true, headerheight=0.1\textheight, headershape=roundedright, headerfont=\Large\bf\textsc, linewidth=2pt 
}
%----------------------------------------------------------------
%	TITLE SECTION 
%----------------------------------------------------------------
{\bf\textsc{Cheat Sheet}\vspace{0.5em}} % Poster title
{\textsc{C h e a t \ \ \ \ \ S h e e t \hspace{12pt}}}
{\textsc{Center for Research Computing \\ University of Pittsburgh \hspace{12pt}}} 

\headerbox{Computation}{name=computation,column=0}{
CRC clusters use the SLURM scheduler. \\
\colorbox[HTML]{cdb87d}{\makebox[\textwidth-2\fboxsep][l]{\bf - Job Management}}
\texttt{sinfo}\hfill \#view info about nodes\\
\texttt{sbatch job.sbatch}\hfill  \#submit a job \\
\texttt{squeue -u \$(whoami)}\hfill \#check job status \\
\texttt{scancel 12345}\hfill \#cancel job with id 12345\\
\texttt{sshare}\hfill \#list the shares of associations\\
\texttt{salloc}\hfill \#get a Slurm job allocation \\
\texttt{sprio 12345}\hfill \#view a job's priority\\
\colorbox[HTML]{cdb87d}{\makebox[\textwidth-2\fboxsep][l]{\bf - Job Status Codes}}
\texttt{PD} \dotfill  Pending \\
\texttt{R}  \dotfill  Running \\
\texttt{CA} \dotfill  Cancelled \\
\texttt{F} \dotfill   Failed \\
\texttt{CF} \dotfill   Configuring \\
\texttt{TO} \dotfill   Timed Out \\
\texttt{PR} \dotfill   Preempted \\
\texttt{NF} \dotfill   Node Failed \\
\texttt{S} \dotfill   Suspended \\
\texttt{CG} \dotfill  Completing \\
\texttt{CD} \dotfill  Completed \\
}

\headerbox{Notes}{name=notes, below=computation, column=0,row=1}{
\begin{itemize}\compresslist
\item In software, superscripts indicate availability over a cluster: h=HTC, s=SMP, g=GPU, o=OPA, items in {\color{red}red} indicate licensed software
\item To print this document on letter size paper use the printer's \textbf{fit to size} option.
\item \LaTeX\: source on git: \texttt{github.com/\\ketancmaheshwari/crc\_cheatsheet}.
\end{itemize}
\colorbox[HTML]{cdb87d}{\makebox[\textwidth-2\fboxsep][l]{\bf - Acronyms}}
\texttt{HTC} \dotfill  High Throughput Computing\\
\texttt{OPA} \dotfill  Omni-Path \\
\texttt{IB FDR} \dotfill  InfiniBand Fourteen Data Rate\\
\texttt{GPU} \dotfill  Graphics Processing Unit \\
\texttt{SMP} \dotfill  Shared Memory Processing \\
\texttt{SSD} \dotfill  Solid State Drive \\
\texttt{NTA} \dotfill  Non-Traditional Architecture \\
\texttt{NVMe} \dotfill  Non-volatile Memory express \\
\texttt{ZFS} \dotfill  Zettabyte File System
}

%------------------------------------------------
% Example SLURM Script
%------------------------------------------------

\headerbox{Example SLURM Scripts}{name=exampleslurm,column=1,span=2,row=0}{
\colorbox[HTML]{cdb87d}{\makebox[\textwidth-2\fboxsep][l]{\bf - for HTC}}
\#!/bin/bash\\
\#SBATCH -N 1 \hfill\# request one core, ensure that all cores are on one machine\\
\#SBATCH --job-name=abf5 \hfill\#name of job: will show up in status output of \texttt{squeu}\\
\#SBATCH --output=abf5.out \hfill\#standard output goes to this file, SLURM will create one if not provided\\
\#SBATCH -t 3-00:00 \hfill\# 3 days walltime in D-HH:MM format\\
\#SBATCH --cpus-per-task=16 \hfill\# Request that ncpus be allocated per process.\\
\#SBATCH --mem=230g \hfill\# Memory requested for all cores (see also --mem-per-cpu)\\
module load HISAT2/2.0.3-beta\\
hisat2-build -p 16 --ss hg38.ss --exon hg38.exon hg38.fa hg38\_tran\\

\colorbox[HTML]{cdb87d}{\makebox[\textwidth-2\fboxsep][l]{\bf - for SMP/MPI/OPA}}
\#!/bin/bash\\
\#SBATCH –job-name=p7n64\\
\#SBATCH --nodes=2 \hfill\#number of nodes requested\\
\#SBATCH --tasks=56\\
\#SBATCH --cpus-per-task=1\\
\#SBATCH --partition=opa \hfill\# partition name is required \\
\#SBATCH --cluster=dist \hfill\# cluster name is required \\
\#SBATCH --mail-user=shs159@pitt.edu \hfill\#send email to this address if ...\\
\#SBATCH --mail-type=END,FAIL \hfill\# ... job ends or fails\\
\#SBATCH --time=02:02:00 \hfill\# 2 hours and 2 mins walltime in hh:mm:ss format\\
\#SBATCH --reservation=sam\_4\\
module purge \hfill\#make sure the modules environment is sane \\
module load intel/2017.1.132 intel-mpi/2017.1.132 fhiaims/160328\_3\\
export I\_MPI\_FABRICS\_LIST=tmi:I\_MPI\_FALLBACK=0\\
%cp \$SLURM\_SUBMIT\_DIR/\$SLURM\_JOB\_NAME  \$SLURM\_SCRATCH\\
%cd \$SLURM\_SCRATCH\\
srun ~/bin/dg3d\_explicit\_mpi.mpi \$HOME/p6n32/run.inpt\\
%mkdir \$SLURM\_SUBMIT\_DIR/\$SLURM\_JOB\_NAME\\

\colorbox[HTML]{cdb87d}{\makebox[\textwidth-2\fboxsep][l]{\bf - for GPU}}
\#!/bin/bash \\
\#SBATCH --job-name=gputf\\
\#SBATCH --output=gputf.std.out\\
\#SBATCH --error=gputf.std.err \hfill\#standard error goes to this file\\
\#SBATCH --time=00:10:00\\
\#SBATCH --nodes=1\\
\#SBATCH --ntasks=1\\
\#SBATCH --cluster=gpu\\             
\#SBATCH --partition=gpu\\             
\#SBATCH --gres=gpu:4 \hfill\#use all four GPU devices on this node\\
module purge\\
\#if your executable was built with CUDA, load the CUDA module:\\
module load cuda/8.0.44\\
module load python/anaconda2.7-4.2.0\\
module load gcc/5.4.0\\
python tensorflowtest.py
}

\headerbox{Troubleshoot}{name=troubleshoot,column=3,row=0}{
\colorbox[HTML]{cdb87d}{\makebox[\textwidth-2\fboxsep][l]{\bf - ssh connection}}
\begin{itemize}\compresslist
	\item If you are on a wireless network, make sure the \textbf{VPN} connection is established.
	\item Use \texttt{ping} to check network connectivity to host, eg. \texttt{ping htc.sam.pitt.edu}
	\item Use ssh in verbose mode with \texttt{-v} to identify possible causes, eg. \texttt{ssh -v h2p.crc.pitt.edu}
\end{itemize}
\colorbox[HTML]{cdb87d}{\makebox[\textwidth-2\fboxsep][l]{\bf - jobs submission}}
\begin{itemize}\compresslist
	\item Sanity test the environment by submitting a simple job, eg.\\
\texttt{
\#!/bin/bash\\
\#SBATCH --output=test.out\\ 
\#SBATCH --error=test.err\\ 
\#SBATCH -t 00:10:00\\
srun echo "Hello \$(hostname)"
}
	\item Check output of \texttt{squeue -t PD} and \texttt{smap} 
\end{itemize}
\colorbox[HTML]{cdb87d}{\makebox[\textwidth-2\fboxsep][l]{\bf - software access}}
\begin{itemize}\compresslist
	\item Check if the software is available using \texttt{module spider} and \texttt{module avail}
	\item Check if the module is loaded with \texttt{module list}
\end{itemize}
%\colorbox[HTML]{cdb87d}{\makebox[\textwidth-2\fboxsep][l]{\bf - data access}}

}
%----------------------------------------------------------------
%	REFERENCES  {name=objectives,column=0,row=0}
%----------------------------------------------------------------
%\headerbox{bb}{name=references,column=1,row=0}{}
%----------------------------------------------------------------
%	FUTURE RESEARCH
%----------------------------------------------------------------
%\headerbox{aa}{name=futureresearch,column=1,row=0}{}
%----------------------------------------------------------------
%	CONTACT INFORMATION
%----------------------------------------------------------------
%\headerbox{Contact Information}{name=contact,column=2,span=2,row=0}{}
%----------------------------------------------------------------
\end{poster}
\end{document}


SMPStandard
        24 nodes of 12-core Xeon E5-2643v4 3.40 GHz (Broadwell)
        256 GB RAM
        256 GB SSD & 1 TB SSD
        10GigE
SMPspecialty
        2 nodes of 12-core Xeon E5-2643v4 3.40 GHz (Broadwell)
            256 GB RAM
            256 GB SSD & 3 TB SSD
            10GigE
        2 nodes of 12-core Xeon E5-2643v4 3.40 GHz (Broadwell)
            512 GB RAM
            256 GB SSD & 3 TB SSD
            FDR Infiniband
        1 node of 12-core Xeon E5-2643v4 3.40 GHz (Broadwell)
            256 GB RAM
            256 GB SSD & 6 TB NVMe
            GigE           
 HTC
        20 nodes of 16-core Intel Xeon E5-2630v3, 2.4GHz (Haswell-EP)
        256 GB RAM
        256 GB SSD
        FDR Infiniband
 MPIOP
        96 nodes of 28-core Intel Xeon E5-2690 2.60 GHz (Broadwell)
        64 GB RAM/node
        256 GB SSD
        100 Gb Omni-Path
  MPIIB
        32 nodes of 20-core Haswell (E5-2660 v3) 2.6 GHz (Haswell)
        128 GB RAM/node
        256 GB SSD
        FDR InfiniBand
    NTA
	        7 nodes with 4 NVIDIA Titan X GPGPUs/node
	        8 nodes with 4 NVIDIA GeForce GTX 1080 GPGPUs/node
	        1 node with 2 NVIDIA K40 GPGPUs
	         8 nodes of Intel KNL
	     Legacy
	 	20 nodes with 16-core Intel Ivy Bridge (E5-2650v2) 2.6 GHz, 64 GB of memory, 1 TB HDD, and FDR IB.
	 	24 nodes with 16-core Intel Sandy Bridge (E5-2650) 2.6 GHz, 128 GB of memory, 1TB HDD, and FDR IB.
	 	82 nodes with 16-core Intel Sandy Bridge (E5-2670) 2.6 GHz. 36 have 32 GB of RAM,1 TB HDD, connected by FDR IB. 36 have 64 GB of RAM, 1 TB HDD, connected by FDR IB. 8 have 64 GB of RAM, 2 TB HDD, connected by GigE. 2 have 128 GB of RAM, 3 TB HDD, connected by FDR IB.
	 	23 nodes with 48-core AMD Magny Cours (6172) 2.1 GHz CPU. 2 nodes have 256 GB RAM, 18 have 128 GB RAM, and 3 have 64 GB RAM.
	 	44 nodes with 12-core Intel Westmere (X5650) 2.67 GHz CPU and 48 GB RAM
	 	110 nodes with 8-core Intel Nehalem CPU (2.93 GHz X5570, 2.67 GHz X5550, and 2.27 GHz L5520). 8 have 48 GB RAM, 56 have 12 GB RAM and 46 have 24 GB RAM.
	 	54 nodes with 64-core AMD Interlagos (Opteron 6276) 2.3 GHz, QDR IB, 2TB HDD. 18 nodes have 256 GB RAM. 36 nodes have 128 GB of RAM.
	 	4 nodes with 4 NVIDIA Tesla C2050 GPGPUs
	 	4 nodes with 4 NVIDIA GTX Titan GPGPUs
	 	1 node with 8-core Intel Sandy Bridge (E5-2643), 128 GB RAM, 3TB SSD
	 1 node with 12-core Haswell (E5-2620 v3) 2.4 GHz, 128 GB RAM, 2 x 250 GB HDD, 2 x 800 GB SSD

